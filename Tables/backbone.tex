\begin{table}[ht]
    \centering
    \begin{tabular}{c|c|c|c|c|c|c} 
     \hline
    \multirow{2}{*}{}  & \multicolumn{2}{c|}{GoogLeNet} & \multicolumn{2}{c|}{VGG-16} & \multicolumn{2}{c}{AlexNet} \\ \cline{2-7}
    ~&RMSE&DSSIM&RMSE&DSSIM&RMSE&DSSIM\\
    \hline
    (a) & 0.1304 & 0.0656 & 0.1269 & \textbf{0.0642} & 0.1638 & 0.0799 \\
    (b) & \textbf{0.1292} & \textbf{0.0656} & \textbf{0.1303} & 0.0661 & 0.1318 & 0.0717 \\
    (c) & 0.1329 & 0.0696 & 0.1336 & 0.0718 & \textbf{0.1239} & \textbf{0.0686} \\
    \hline 
    \end{tabular}
    \caption[不同特征提取网络的对比]{
    \label{table:backbone}
    使用不同特征提取网络的结果比较,评价指标采用RMSE和DSSIM。 对于每种来引入的网络结构和参数,使用三种方式来评估网络性能:(a)从头开始训练整个网络(train from scratch); (b)从预训练模型中微调(fine-tune)。 (c)固定特征提取网络的预训练参数,仅训练照明估计网络(freeze)。 结果表明,在不同网络中处理特征提取网络的最佳方式是不同的。 这取决于采用的骨干模型以及训练数据的大小。 基于此表格,我们使用固定参数的AlexNet作为特征提取网络。
    }
\end{table}