%---------------------------------------------------------------------------%
%->> 封面信息及生成
%---------------------------------------------------------------------------%
%-
%-> 中文封面信息
%-
\confidential{}% 密级:只有涉密论文才填写
\schoollogo{scale=0.095}{ucas_logo}% 校徽
\title{基于深度学习的场景光照估计研究}% 论文中文题目
\author{}% 论文作者
\advisor{}% 指导教师:姓名 专业技术职务 工作单位
\advisors{中国科学院软件研究所}% 指导老师附加信息 或 第二指导老师信息
\degree{硕士}% 学位:学士、硕士、博士
\degreetype{工学}% 学位类别:理学、工学、工程、医学等
\major{计算机应用技术}% 二级学科专业名称
\institute{中国科学院软件研究所}% 院系名称
\date{2019~年~6~月}% 毕业日期:夏季为6月、冬季为12月
%-
%-> 英文封面信息
%-
\TITLE{Deep Scene Illumination Estimation}% 论文英文题目
\AUTHOR{}% 论文作者CHENG Dachuan
\ADVISOR{}% 指导教师Supervisor: Professor Chen Yanyun
\DEGREE{Master}% 学位:Bachelor, Master, Doctor。封面格式将根据英文学位名称自动切换,请确保拼写准确无误
\DEGREETYPE{Engineering}% 学位类别:Philosophy, Natural Science, Engineering, Economics, Agriculture 等
\THESISTYPE{thesis}% 论文类型:thesis, dissertation
\MAJOR{Computer Graphics}% 二级学科专业名称
\INSTITUTE{Institute of Software, Chinese Academy of Sciences}% 院系名称
\DATE{June, 2019}% 毕业日期:夏季为June、冬季为December
%-
%-> 生成封面
%-
\maketitle% 生成中文封面
\MAKETITLE% 生成英文封面
%-
%-> 作者声明
%-
\makedeclaration% 生成声明页
%-
%-> 中文摘要
%-
\chapter*{摘\quad 要}\chaptermark{摘\quad 要}
\setcounter{page}{1}% 开始页码
\pagenumbering{Roman}% 页码符号
光照估计是计算机视觉和计算机图形中的基础问题之一,该问题意图从图片中恢复出场景真实的光照分布。光照估计问题与其它多种视觉和图形学问题息息相关。本文通过对已有光照估计方法的调研和分析,提出了一种基于深度学习的场景光照估计方法。该方法使用智能设备前后置相机拍摄的图片作为输入,预测出使用球形谐波函数近似的场景光照分布。为此本文搭建了一个端到端的深度卷积神经网络,并创新性地提出了一个损失函数——渲染损失(Render Loss),用于监督神经网络的训练,提高光照估计预测效果。本文的方法在可视结果和数值结果上都超过了目前最先进的光照估计算法,而且该方法真实场景中的效果也非常理想。此外,本文还构建了一个大规模的高动态范围全景数据集,用于训练和测试光照估计网络。最后本文通过在光照估计数据集和光照估计网络上的大量实验,深入探究了数据规模、数据多样性、神经网络结构、损失函数等多个因素对于光照估计效果的具体影响。

\keywords{光照估计,深度学习,球形谐波函数}% 中文关键词
%-
%-> 英文摘要
%-
\chapter*{Abstract}\chaptermark{Abstract}% 摘要标题
Illumination estimation is an essential problem in computer vision and graphics. It is closely related to many vision or graphic problems. In this paper, a learning based method is proposed to recover scene illumination represented as spherical harmonic (SH) functions by pairwise photos from rear and front cameras on mobile devices. An end-to-end deep convolutional neural network (CNN) structure is designed to process images on symmetric views and predict SH coefficients. We introduce a novel Render Loss to improve the rendering quality of the predicted illumination. Experiments show that our model produces visually and quantitativelysuperior results compared to the state-of-the-arts. A high quality high dynamic range (HDR) panoramic imagedataset is developed for training and evaluation. Moreover, a series of experiment is conducted to explore how the data size, data diversity, network structure, loss function and other modules affect the illumination  estimation results.

\KEYWORDS{Illumination Estimation, Deep Learning, Spherical Harmonic Lighting}% 英文关键词
%---------------------------------------------------------------------------%
