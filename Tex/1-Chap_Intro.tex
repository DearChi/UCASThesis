\chapter{绪论}\label{chap:introduction}

\section{选题的背景及意义}
%% 学科背景

%% 问题定义与介绍
光照估计(又称光照分布估计)是从已知的彩色图像信息中,预测、估计、恢复出整个场景的光照分布。
该问题的输入通常是若干张彩色图片,或者是一段视频,有时几何或材质信息也会作为先验知识来辅助估计光照。
场景的光照分布是指场景中各个方向的光照的颜色和强度。
较为常见的光照分布表示方法包括
高动态范围(High Dynamic Range,HDR)全景图、
点面光源的有限集、球形谐波(Spherical Harmoins,SH)表示、小波函数表示等。
其中精度最高、使用比较广泛的是HDR全景图像,在实时渲染领域使用较多的则是光照的SH表示。 

%% 研究意义、应用场景
光照估计作为计算机图形学和计算机视觉的基础问题之一,有着广泛的实际应用场景。
例如:基于图像的渲染(Image Based Rendering,IBR)、
增强现实(Augmented Reality,AR)、电影后期制作、真实感虚实交互等。
图\ref{fig:demo-problem-define}展示了光照估计的应用之一。
光照估计也与学科中的许多其它问题息息相关。
例如:双向反射分布函数(BRDF)估计、场景几何重构、本征信息提取、图像增强,等等。
而高质量的光照估计结果,通常能够为这些问题的解决带来很大的帮助。

\begin{figure}[!htbp]
    \centering
    \includegraphics[width=0.40\textwidth]{Img/ucas_logo.pdf}

    \bicaption[光照估计的应用]
    {光照估计的应用之一。使用单张图片估计场景的光照,并利用估计的光照渲染一个新的物体合成到图像中。
    可以看出使用估计光照渲染后的3D物体,与场景中的已有物体在视觉上较为一致。}
    {3D rendering under the predicted illumination.
    visual effect of the 3D rendering is in line with the original image.}
    
    \label{fig:demo-problem-define}
\end{figure} 

%% 难点
从有限的图像信息估计出整个场景的光照分布是一个复杂的问题。
首先,图像的视野范围比较有限,
例如一张视场角(FOV)为60°的照片所拍摄到的区域,在其对应的全景图中占比不足6\%。
此外,一幅图片是光照分布、场景几何结构、物体材质、摄相机参数等多个单位之间的复杂交互结果
(公式~\ref{eq:complex-interaction-result})。
\begin{equation} \centering \label{eq:complex-interaction-result}
Image = ComplexInteraction(Light, Geometry, Material, Camera)\end{equation}
通过公式\ref{eq:complex-interaction-result}可以看出,
在其它三个信息未知的情况下,从图像(Image)反推出光照(Light)是一个严重的不适定(ill-posed)问题。
不仅如此,在不同的条件下拍摄的彩色图像可能存在很多误差。
例如图像中的过曝光/欠曝光区域、相机畸变、不正确的白平衡等。
这些都会对光照估计造成一定程度的干扰,增加光照估计的难度。

%% 传统方法 + 局限性
研究者们在光照估计问题上开展了一系列的研究。Debevec\cite{debevec1998rendering}
首次提出可以通过拍摄不同曝光下的镜面球体,来获取一张高动态范围的全景图片。
\todo{Related limited methods.} 
可以看出,这些方法通常无法从图像直接估计光照,而是需要额外的先验信息、操作步骤、模型约束和假设。

%% 深度学习方法 + 局限性
近年来,深度学习在多种计算机视觉问题上大放异彩,用于分割、检测、标识、分类的神经网络层出不穷。
因此一些研究者尝试将深度学习应用在光照估计问题当中。
\todo{Related learning based method.}
不过,目前已有的深度学习方法也有一定的局限性。训练一个鲁棒的神经网络往往需要大量的数据。
目前用于光照估计问题的数据集比较有限,主要包括:
大规模的低动态范围全景数据集(SUN360等)和中小规模特定场景的高动态范围全景数据集(Laval Indoor等)。
这些数据集在规模和质量上很难同时到达训练深度神经网络的要求。

%% 研究目标和研究方法, 在前人工作的基础上,对网络、loss、优化,数据 展开研究
在这样的背景下,本文构建了一个用于光照估计的数据集。
数据集具有一定的规模和质量,它不仅能被用来训练更加鲁棒的光照估计网络,也可以被应用到其它多种相关的深度学习问题当中。
此外,本文探索了基于深度学习网络的光照估计方法,对网络结构,损失函数,光照表示等多个模块进行了深入的研究。

\section{国内外研究现状}
\subsection{光照的表示与获取}
\subsection{传统光照估计方法及其扩展}
\subsection{深度学习}
\subsection{基于深度学习的光照估计方法}
\subsection{光照估计相关数据集}