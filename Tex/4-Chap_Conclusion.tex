\chapter{总结与展望}
\section{本文工作总结}
光照估计(又称光照分布估计)是从已知的彩色图像信息中,预测、估计、恢复出整个场景的光照分布。场景的光照分布是指场景中各个方向上光照的颜色和强度。该问题的输入通常是若干张彩色图片,或者是一段视频,有时已知的几何或材质信息也回被用来辅助估计光照。光照估计作为计算机图形学和计算机视觉的基础问题之一,有着广泛的实际应用场景。例如:基于图像的渲染(Image Based Rendering,IBR)、增强现实(Augmented Reality,AR)、电影后期制作、真实感虚实交互等。图\ref{fig:demo-problem-define}展示了光照估计的应用之一。光照估计也与这两个学科中的许多其它问题息息相关。例如:双向反射分布函数(BRDF)估计、场景几何重构、本征信息提取、图像增强,等等。高质量的光照估计结果通常能够为这些问题的解决带来很大的帮助。

从有限的图像信息估计出整个场景的光照分布是一个复杂的问题。一方面,图像的视野范围比较有限,例如一张视场角(FOV)为60°的照片所拍摄到的区域,在其对应的全景图中占比不足6\%。另一方面,一幅图片是光照分布、场景几何结构、物体材质、摄相机参数等多个单位之间的复杂交互结果,从图像反推出光照是一个严重的不适定(ill-posed)问题。不仅如此,在不同的条件下拍摄的彩色图像可能存在很多误差。例如图像中的过曝光/欠曝光区域、相机畸变、不正确的白平衡等,这些都会对光照估计问题造成一定程度的干扰,增加光照估计的难度。

为了降低问题难度,传统方法常常增加输入信息的数量或简化光照的表示模型。其中,一部分工作尝试使用更多的输入信息辅助估计场景光照。例如深度信息、几何信息、多张图片、先验知识、用户标记等等。这类方法通常依赖特殊的探针、特殊的拍摄设备、辅助信息,均具有一定的限制。另一部分工作通过使用低维的光照表示模型来简化光照估计问题,例如使用球形谐波函数(SH)来拟合场景光照、使用小波函数近似场景光照、使用若干个点光源的集合近似场景光照、使用基于物理的室外光照模型等等。可以看出,无论是增加输入信息还是使用简化的光照估计模型,传统光照估计方法都具有很大的局限性。

近年来,深度学习在多种计算机视觉问题上大放异彩,用于分割、检测、标识、分类的神经网络层出不穷。一些研究者尝试将深度学习应用在光照估计问题当中。其中Hold-Geoffroy等人\cite{hold2017deep}和Gardner\cite{gardner2017learning}等人的工作是应用深度学习估计光照的最先进方法。它们在大规模数据集上训练了一个深度卷积神经网络,分别用于室内和室外的场景光照估计,能达到较好的效果。不过,目前已有的深度学习方法也有一定的局限性。训练一个鲁棒的神经网络往往需要大量的数据,而目前用于光照估计问题的数据集比较有限,主要包括:大规模的低动态范围全景数据集(SUN360\cite{xiao2012recognizing}等)和中小规模特定场景的高动态范围全景数据集(Laval Indoor等\cite{gardner2017learning})。这些数据集在规模和质量上很难同时到达训练深度神经网络的要求。

在这样的背景下,本文在基于深度学习的光照估计问题上的两个主要方向上进行了细致的研究。其一是构建了一个具有一定规模和质量的光照估计数据集。数据集由高动态范围全景图组成,这样的数据集不仅能被用来训练更加鲁棒的光照估计网络,也可以被应用到其它多种相关的深度学习问题当中。其二是在已有数据集和本文构建的数据集基础上,深入探索基于深度学习的光照估计方法,对其中的网络结构,网络参数,损失函数,光照表示等多个模块进行了细致的对比和研究。作为总结,将本文的主要工作内容和创新贡献罗列如下:
\begin{itemize}
    \item 深入调研了全景图以及高动态范围全景图的获取步骤、投影方法、存储方式等,使用全景相机和曝光融合工具构建了一个用于光照估计的,大规模高质量的HDR全景数据集,并通过实验验证了该数据集相对于其它数据集在光照估计问题中的优越性。
    \item 通过详细的实验分析了HDR全景数据集的规模和多样性对于光照估计神经网络的影响,证明了丰富的数据多样性和较大的数据规模能够为基于深度学习的光照估计结果带来有效提升。
    \item 首次提出使用视角相对的图片作为光照估计的输入,这两张图片多由相机的前后置摄像头拍摄。使用前后摄像头同时拍摄两张照片不仅不会增加获取图片的步骤,还可以极大地降低光照估计的难度。这种方法对于光照估计在智能设备中的应用来说有着很大的实际意义。
    \item 构建了一个基于深度学习的光照估计网络模型,该网络模型使用两张图片作为输入,估计预测出场景中光照对应的球形谐波系数。已有的实验结果表明,该模型在光照估计问题中非常有效,目前已经取得了超过state-of-the-art的结果。
    \item 提出了一个新的损失函数 - Render Loss,该损失函数巧妙地利用了SH的特性,将部分渲染过程置于神经网络中,进而在训练中对渲染结果进行监督,指导网络的优化与调整。Render Loss的实现算法简洁但非常有效,极大地提高了光照估计的表现。
    \item 深入探索了本文所提的光照估计深度学习网络。对于光照估计网络结构和训练过程中各个模块进行了十分详尽的实验和分析,对比了不同的网络结构、网络参数和损失函数对光照估计结果的影响。
\end{itemize}

\section{未来工作展望}
本文构建的数据集和光照估计网络取得了较好的效果。在进行光照估计相关实验时,发现了可以进一步拓展、提高、优化的部分。由于这些内容或与本文工作相关性不大、或受限于目前的硬件和技术限制无法开展实验、或由于实验周期较长难以在有限时间内完成,目前还没有被包含在本文的工作中。这些内容在满足研究条件后,都可能成为一些新的、独立的研究课题。

在光照估计数据集方面,本文构建的光照估计数据集包含了近千张HDR全景图像。这个数量还可以增加。通过第2章的实验可以发现,在训练用于光照估计的深度网络时,增加数据的多样性比单纯的增加数据更加有用,这可以作为后期扩充数据时的主要指导和依据。

在输入信息方面,本文的方法使用现代智能设备前后相机拍摄的两幅图片作为输入。在实际使用中,人脸可能会经常出现在前置摄像头中。本文所提出的方法中没有针对人脸的部分,因此为了避免人脸所占区域过多,需要用户让手机或者头部做一下水平偏移。不过,出现在前置摄像头中的人脸并不能完全视为光照估计问题的一个阻碍。一些光照估计方法专门使用人脸作为标志物辅助估计场景的光照。这些工作为基于前后摄像头估计光照的方法提供了一个新的思路——可以将基于人脸的光照估计方法嵌入到本文的方法中,从前置相机的人脸和后置相机的普通图片中共同恢复场景光照。

在光照表示方面,球形谐波函数本身具有一定的局限性,它在表示光照时难以处理非常高频的光照细节,在渲染时对镜面材质不太友好。一些工作使用自编码器(Auto-Encoder)对光照进行建模,这种方式仍然会丢失一部分精度信息,并且非常依赖用于构建自编码器的训练数据。如何在尽可能不损失精度的情况下,表示兼具高频和低频的光照信息也是一个值得研究的方向

在深度学习模型方面,模型的性能也是一个需要考虑的因素。本文模型使用的网络结构是AlexNet,虽然这是在目前光照估计数据集上表现较好的网络,但Alexnet本身参数量较大,计算需求高,难以在普通设备中部署和应用。轻量和高效是网络模型能够用于移动设备中关键因素,因此在训练数据得到扩充之后,需要考虑如何使用更加轻量和高效的模型。

此外,视频也可以作为光照估计问题的输入,视频往往包含了更丰富的信息。传统方法使用视频估计光照时通常能获得较好的效果,但目前还没有使用深度学习从视频中估计光照的方法,这也是一个值得探索的研究方向。